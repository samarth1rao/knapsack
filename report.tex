\documentclass[12pt,a4paper]{article}

% Packages
\usepackage[utf8]{inputenc}
\usepackage[margin=1in]{geometry}
\usepackage{graphicx}
\usepackage{amsmath}
\usepackage{amssymb}
\usepackage{algorithm}
\usepackage{algpseudocode}
\usepackage{hyperref}
\usepackage{cite}
\usepackage{caption}
\usepackage{subcaption}
\usepackage{booktabs}
\usepackage{listings}
\usepackage{xcolor}

% Code listing style
\lstset{
    basicstyle=\ttfamily\small,
    breaklines=true,
    frame=single,
    numbers=left,
    numberstyle=\tiny,
    commentstyle=\color{gray},
    keywordstyle=\color{blue},
    stringstyle=\color{red}
}

% Hyperref setup
\hypersetup{
    colorlinks=true,
    linkcolor=blue,
    filecolor=magenta,      
    urlcolor=cyan,
    citecolor=blue,
}

% Title page information
\title{\textbf{A Modified Genetic Approach to the 0/1 Knapsack Problem} \\ 
\Large{Algorithm Analysis and Design: Course Project}}
\author{
    Ishan Bansal \and
    Shuban Biswas \and
    Akshat Gupta \and
    Siddharth Choudhary \and
    Samarth Rao \\
}

\begin{document}

\maketitle

% Abstract on title page
\begin{abstract}
\noindent
[Write a 150-250 word summary of your project here. Include: (1) the problem you are solving, (2) the algorithms you implemented, (3) your methodology, and (4) key findings from your experiments. Keep it concise and informative.]
\end{abstract}

\thispagestyle{empty}

\newpage
\tableofcontents
\newpage

% 1. Introduction
\section{Introduction}
\label{sec:introduction}

[Define the problem(s) you are solving in this section. Discuss the real-world relevance and applications of these problems. State the objectives of your project clearly.]

\subsection{Problem Statement}
[Provide a formal definition of the problem(s) you are addressing.]

\subsection{Motivation}
[Explain why this problem is important and where it appears in real-world applications.]

\subsection{Project Objectives}
[List the specific goals of your project. What algorithms will you implement? What comparisons will you make?]

% 2. Algorithm Descriptions
\section{Algorithm Descriptions}
\label{sec:algorithms}

[For each algorithm implemented, create a subsection. Provide a clear theoretical explanation of how the algorithm works.]

\subsection{Algorithm 1: [Algorithm Name]}
\label{subsec:algo1}

\subsubsection{Overview}
[Provide a high-level description of the algorithm and its purpose.]

\subsubsection{Theoretical Foundation}
[Explain the theoretical basis of the algorithm. Include any relevant mathematical formulations.]

\subsubsection{Pseudocode}
\begin{algorithm}
\caption{[Algorithm Name]}
\label{alg:algo1}
\begin{algorithmic}[1]
\Procedure{AlgorithmName}{$input$}
    \State [Your pseudocode here]
    \State ...
    \Return $result$
\EndProcedure
\end{algorithmic}
\end{algorithm}

\subsubsection{Complexity Analysis}
\paragraph{Time Complexity:} [Provide detailed time complexity analysis with justification. Include best, average, and worst-case scenarios if applicable.]

\paragraph{Space Complexity:} [Provide detailed space complexity analysis with justification.]

\subsection{Algorithm 2: [Algorithm Name]}
\label{subsec:algo2}

[Repeat the same structure as Algorithm 1]

\subsection{Algorithm 3: [Algorithm Name]}
\label{subsec:algo3}

[Add more algorithm subsections as needed]

% 3. Implementation Details
\section{Implementation Details}
\label{sec:implementation}

[Discuss your key design choices in this section.]

\subsection{Programming Language and Environment}
[State which programming language(s) you used and justify your choice.]

\subsection{Data Structures}
[Explain the data structures you used for each algorithm and why you chose them.]

\begin{itemize}
    \item \textbf{Algorithm 1:} [Data structures used and rationale]
    \item \textbf{Algorithm 2:} [Data structures used and rationale]
    \item \textbf{Algorithm 3:} [Data structures used and rationale]
\end{itemize}

\subsection{Implementation Challenges}
[Discuss the most significant challenges you faced during implementation and how you overcame them.]

\subsection{Code Organization}
Our repository is organized into modular directories, with each algorithm implementation isolated in its own folder. The primary structure consists of an \texttt{algorithms/} directory containing 12 distinct algorithm implementations, a \texttt{simulation/} directory for benchmarking and visualization, and supporting documentation files.

Each algorithm folder contains:
\begin{itemize}
    \item A C++ implementation file (\texttt{.cpp})
    \item A dedicated \texttt{README.md} with algorithm-specific documentation
    \item Reference materials where applicable (research papers, example implementations)
\end{itemize}

The \texttt{bin/} subdirectory stores compiled executables, and a centralized \texttt{Makefile} manages the build process for all algorithms. The \texttt{simulation/} directory contains Python scripts for running experiments, generating performance metrics, and visualizing results.

\paragraph{Repository Structure:}
\begin{lstlisting}[language=bash]
knapsack/
├── ARCHITECTURE.md
├── algorithms/
│   ├── 01-bruteforce/
│   │   ├── bruteforce.cpp
│   │   └── README.md
│   ├── 02-memoization/
│   │   ├── memoization.cpp
│   │   └── README.md
│   ├── 03-dynamicprogramming/
│   │   ├── dynamicprogramming.cpp
│   │   └── README.md
│   ├── 04-branchandbound/
│   │   ├── branchandbound.cpp
│   │   └── README.md
│   ├── 05-meetinthemiddle/
│   │   ├── algo.cpp
│   │   └── README.md
│   ├── 06-greedyheuristic/
│   │   ├── greedyheuristic.cpp
│   │   └── README.md
│   ├── 07-randompermutation/
│   │   ├── randompermutation.cpp
│   │   └── README.md
│   ├── 08-efficientalgo/
│   │   ├── efficientalgo.cpp
│   │   ├── README.md
│   │   └── reference/
│   ├── 09-billionscale/
│   │   ├── algo.cpp
│   │   └── README.md
│   ├── 10-geneticalgorithm/
│   │   ├── geneticalgorithm.cpp
│   │   ├── README.md
│   │   └── reference/
│   ├── 11-customalgorithm/
│   │   ├── customalgorithm.cpp
│   │   ├── README.md
│   │   └── reference/
│   ├── 12-customtestbed/
│   │   ├── customtestbed.cpp
│   │   └── README.md
│   ├── bin/
│   ├── Makefile
│   └── README.md
└── simulation/
    ├── simulation.py
    ├── requirements.txt
    └── readme.txt
\end{lstlisting}

% 4. Experimental Setup
\section{Experimental Setup}
\label{sec:experimental}

\subsection{Hardware and Software Environment}
\begin{itemize}
    \item \textbf{Processor:} [e.g., Intel Core i7-9700K @ 3.6GHz]
    \item \textbf{RAM:} [e.g., 16GB DDR4]
    \item \textbf{Operating System:} [e.g., Ubuntu 22.04 LTS]
    \item \textbf{Programming Language:} [e.g., Python 3.11.5]
    \item \textbf{Libraries Used:} [List standard libraries used]
\end{itemize}

\subsection{Datasets}
[Describe the datasets you used for testing and benchmarking.]

\subsubsection{Synthetic Data}
[If you generated synthetic data, explain how it was generated and what parameters were varied.]

\subsubsection{Real-World Data}
[If you used real-world datasets, cite the sources and explain their characteristics.]

\subsection{Metrics}
[List and explain the metrics you used to evaluate algorithm performance:]
\begin{itemize}
    \item \textbf{Wall-Clock Time:} [Explanation]
    \item \textbf{Memory Usage:} [Explanation]
    \item \textbf{Solution Quality:} [If applicable]
    \item \textbf{Number of Operations:} [e.g., comparisons, swaps]
\end{itemize}

% 5. Results and Analysis
\section{Results and Analysis}
\label{sec:results}

[This is the most important section. Present your results clearly using graphs, charts, and tables. Compare empirical performance against theoretical complexity.]

\subsection{Experimental Results}

\subsubsection{Performance Comparison: [Metric 1]}

\begin{figure}[h]
    \centering
    \includegraphics[width=0.8\textwidth]{figures/performance_comparison.png}
    \caption{Performance comparison of algorithms based on [metric]. The x-axis shows [input size/parameter], and the y-axis shows [metric measured].}
    \label{fig:perf_comparison}
\end{figure}

[Discuss what Figure~\ref{fig:perf_comparison} shows. Explain trends, outliers, and interesting observations.]

\subsubsection{Theoretical vs. Empirical Analysis}

\begin{table}[h]
\centering
\caption{Comparison of theoretical and empirical complexities}
\label{tab:complexity_comparison}
\begin{tabular}{@{}lccc@{}}
\toprule
\textbf{Algorithm} & \textbf{Theoretical Time} & \textbf{Empirical Time} & \textbf{Match?} \\ \midrule
Algorithm 1 & $O(n \log n)$ & $\sim n \log n$ & Yes \\
Algorithm 2 & $O(n^2)$ & $\sim n^2$ & Yes \\
Algorithm 3 & $O(n)$ & $\sim n$ & Yes \\ \bottomrule
\end{tabular}
\end{table}

[Discuss Table~\ref{tab:complexity_comparison}. Explain how you verified the empirical complexity matches the theoretical predictions.]

\subsection{Detailed Analysis}

\subsubsection{Algorithm 1 Analysis}
[Deep dive into the performance of Algorithm 1. Why does it perform the way it does? How does it compare to other algorithms?]

\subsubsection{Algorithm 2 Analysis}
[Repeat for Algorithm 2]

\subsubsection{Algorithm 3 Analysis}
[Repeat for Algorithm 3]

\subsection{Cross-Algorithm Comparison}
[Compare all algorithms directly. When does each algorithm perform best? What are the trade-offs?]

\subsection{Discussion}
[Discuss why you see these results. Connect back to theory. Explain any unexpected findings or deviations from theoretical predictions.]

% 6. Conclusion
\section{Conclusion}
\label{sec:conclusion}

\subsection{Summary of Findings}
[Summarize the key findings from your project. What did you learn about the algorithms?]

\subsection{Limitations}
[Discuss limitations of your implementation, experimental setup, or analysis. What could be improved?]

\subsection{Future Work}
[Suggest potential improvements or extensions to your project. What would you do differently? What additional experiments would be valuable?]

% Bonus Disclosure
\section*{Bonus Components}
\addcontentsline{toc}{section}{Bonus Disclosure}

[Clearly specify which components of your project should be evaluated for bonus marks. Be specific about algorithms, metrics, or analysis techniques that go beyond the basic requirements.]

\textbf{Bonus Components:}
\begin{itemize}
    \item \textbf{Additional Algorithm:} [e.g., Algorithm X - advanced variant]
    \item \textbf{Advanced Metrics:} [e.g., Cache miss rate analysis]
    \item \textbf{Extended Analysis:} [e.g., Probabilistic analysis of randomized algorithm]
\end{itemize}

% References
\bibliographystyle{plain}
\bibliography{references}

% If not using BibTeX, use this format:
% \begin{thebibliography}{99}
% 
% \bibitem{cormen2009}
% Cormen, T. H., Leiserson, C. E., Rivest, R. L., \& Stein, C. (2009). 
% \textit{Introduction to Algorithms} (3rd ed.). MIT Press.
% 
% \bibitem{knuth1997}
% Knuth, D. E. (1997). 
% \textit{The Art of Computer Programming, Volume 1: Fundamental Algorithms} (3rd ed.). 
% Addison-Wesley.
% 
% \end{thebibliography}




\end{document}